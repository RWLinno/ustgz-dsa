
\section{Related Works}
\label{sec:relate}
\textbf{Climate Forecasting.} There are two primary approaches for climate forecasting: the physical and statistical approaches. The physical approach, represented by Numerical Weather Prediction (NWP) methods~\cite{bauer2015quiet}, typically requires substantial computational resources to solve simulation equations with boundary conditions. In contrast, statistical approaches, such as GenCast~\cite{price2023gencast}, utilize historical observations to develop large-scale global weather forecasting models that demonstrate superior performance compared to NWP methods. However, both physical and statistical approaches encounter challenges in regional urban climate forecasting, such as UHI effect prediction. At fine-grained scales, physical simulations can become overly complex and time-consuming~\cite{mass1998regional}, while statistical methods are limited to general atmospheric circulation forecasting~\cite{schultz2021can}.


\noindent
\textbf{Multivariate Series Forecasting.} Fine-grain UHI effect forecasting can be approached through multivariate time-series (TS) forecasting or spatio-temporal (ST) forecasting~\cite{shao2024exploring,liu2022contrastive}, the latter of which incorporates spatial information. Deep neural networks, particularly Transformers~\cite{wen2022transformers}, have gained attention in TS forecasting, with models like iTransformer~\cite{liu2023itransformer}, Autoformer~\cite{wu2021autoformer}, and PatchTST~\cite{nie2022time} demonstrating high performance. Fully connected models, such as Dlinear~\cite{zeng2023transformers}, also serve as competitive alternatives.
In spatiotemporal forecasting, STGCN utilizes graph convolution for spatial data processing. STGCN-like models form a broad category of methods capable of addressing both temporal and spatial dependencies~\cite{shao2024exploring,yu2017spatio}, often applied to traffic prediction tasks. Further techniques like STID~\cite{shao2022spatial} offer a simple spatial-temporal identity-attaching approach to capture spatial dependencies without relying on graphs. Recently, there has been a rising trend of building foundation models~\cite{liang2025foundation} for spatio-temporal data science.


\noindent
\textbf{Data-driven Urban Thermal Study.} Contextual environment data like satellite, and street-view imagery are commonly utilized as key data sources to support regional thermal condition studies~\cite{hao2025unlocking,han2024microclimate,hao2025nature}. Existed context data-enhanced techniques often employ regression-based approaches to compute the relation between the local thermal effect and environmental situations. However, the results of these methods are limited to qualitative analysis. \cite{equere2020definition} analysis of the relation between local satellite image and UHI effect and compute the coefficient of determination factor $R^{2}$ of 0.68 between them in San Francisco. \cite{equere2021integration} emphasize the importance of terrain features in UHI spatial inference. \cite{han2024microclimate} incorporate land use data to support the prediction of serval climate situations.